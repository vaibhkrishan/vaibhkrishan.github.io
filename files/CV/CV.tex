\documentclass[a4paper,10pt]{article}

\RequirePackage{color,graphicx}
\usepackage{url,parskip}
\usepackage[usenames,dvipsnames]{xcolor}
\usepackage[big]{layaureo}        %better formatting of the A4 page
% an alternative to Layaureo can be ** \usepackage{fullpage} **
\usepackage{supertabular}         %for Grades
\usepackage{titlesec}         %custom \section

\usepackage{hyperref}
\definecolor{linkcolour}{rgb}{0,0.2,0.6}
\hypersetup{colorlinks,breaklinks,urlcolor=linkcolour, linkcolor=linkcolour}

%CV Sections inspired by: 
%http://stefano.italians.nl/archives/26
\titleformat{\section}{\Large\scshape\raggedright}{}{0em}{}[\titlerule]
\titlespacing{\section}{0pt}{1em}{1em}

\begin{document}

\par{\centering
{\Huge Vaibhav \textsc{Krishan}
}\bigskip\par}

\begin{tabular}{rl}
  Phone: & +91-8454937541 \\
  Personal email: & \href{mailto:vaibhkrishan@gmail.com}{vaibhkrishan@gmail.com} \\ 
  Official emails: & \href{mailto:vaibhavk@imsc.res.in}{vaibhavk@imsc.res.in}\\
  Profile pages: & \href{https://vaibhkrishan.github.io}{Homepage}, 
  \href{https://dblp.org/pid/187/8296.html}{DBLP},
  \href{https://scholar.google.com/citations?user=dVMC44EAAAAJ&hl=en}{Google Scholar},
  \href{https://eccc.weizmann.ac.il/author/1412/">ECCC (IIT Bombay account)</a>, <a href="https://eccc.weizmann.ac.il/author/985/}{ECCC (Personal account)},
  \href{https://orcid.org/0009-0000-0335-1963}{ORCID}
\end{tabular}

\section{Research Positions}

\begin{tabular}{lll}
  {\bf Postdoctoral Fellow} & The Institute of Mathematical Sciences, Chennai & 2024-Present \\
  \textsc{Computer Science} &  & 
\end{tabular}

\section{Education}

\begin{tabular}{lll}
  {\bf Doctoral Student} & Indian Institute of Technology Bombay & 2017-2024 \\
  \textsc{Computer Science} & CPI: 9.09/10 & \\
  \textsc{and Engineering} \\
  \\
  {\bf Bachelor of Technology} & Indian Institute of Technology Bombay & 2009-2013 \\
  \textsc{Computer Science} & CPI: 7.59/10 & \\
  \textsc{and Engineering} & & \\
\end{tabular}

\section{Journal Publications}

\begin{tabular}{p{2.5cm}|p{11cm}l}
  \href{https://link.springer.com/article/10.1007/s00453-021-00915-7}{{\bf Algorithmica 2022}} & A \#SAT Algorithm for Small Constant-depth Circuits with PTF gates \\
  & with Swapnam Bajpai, Deepanshu Kush, Nutan Limaye and Srikanth Srinivasan \\
  & Algorithmica 84, 1132-1162 (2022).
\end{tabular}

\section{Conference Publications}

\begin{tabular}{p{2.5cm}|p{11cm}l}
  \href{https://link.springer.com/chapter/10.1007/978-3-030-79416-3_15}{{\bf CSR 2021}} & Upper Bound for Torus Polynomials \\
  & The 16th International Computer Science Symposium in Russia, CSR 2021. \\\multicolumn{2}{c}{} \\
  \\
  \href{https://drops.dagstuhl.de/opus/volltexte/2018/10101/}{{\bf ITCS 2019}} & A \#SAT Algorithm for Small Constant-depth Circuits with PTF gates \\
  & with Swapnam Bajpai, Deepanshu Kush, Nutan Limaye and Srikanth Srinivasan \\
  & The 10th 10th Innovations in Theoretical Computer Science Conference, ITCS 2019.
\end{tabular}

\section{Preprints}

\begin{tabular}{p{2.5cm}|p{11cm}l}
  \href{https://eccc.weizmann.ac.il/report/2024/074/}{{\bf ECCC}} & Towards \(\mathsf{ACC}\) Lower Bounds using Torus Polynomials \\
                                                                  & with Sundar Vishwanathan \\
  \href{https://eccc.weizmann.ac.il/report/2023/111/}{{\bf ECCC}} & \(\mathsf{MidBit}^+\), Torus Polynomials and Non-classical Polynomials: Equivalences for \(\mathsf{ACC}\) Lower Bounds \\
  \href{https://eccc.weizmann.ac.il/report/2016/155/}{{\bf ECCC}} & Isolation Lemma for Directed Reachability and NL vs. L
  \\
  & with Nutan Limaye
\end{tabular}

\section{Talks and Presentations}

\begin{tabular}{p{2.5cm}|p{11cm}l}
  \href{https://vaibhkrishan.github.io/files/slides/torus.pdf}{Presentation} & Upper Bound for Torus Polynomials \\
  & The 16th International Computer Science Symposium in Russia, CSR 2021. (online) \\\multicolumn{2}{c}{} \\
  \\
  \href{https://vaibhkrishan.github.io/files/slides/ptf.pdf}{Presentation} & A \#SAT Algorithm for Small Constant-depth Circuits with PTF gates \\
  and \href{https://vaibhkrishan.github.io/files/slides/poster.pdf}{Poster} & The 10th Innovations in Theoretical Computer Science Conference, ITCS 2019.
\end{tabular}

\section{Service and Engagement}
\begin{tabular}{p{2.5cm}|p{11cm}l}
  Sub-reviewer & Computational Complexity Conference (CCC) 2024. \\
  Sub-reviewer & Foundations of Software Technology and Theoretical Computer Science (FSTTCS) 2024 \\
  Volunteer & FSTTCS 2019
\end{tabular}

\section{Teaching Assistance}

\begin{itemize}
  \item Awarded best TA of the month for Automata Theory.
  \item Awarded best TA for CS 101.
\end{itemize}

\section{Professional Experience}

\begin{itemize}
  \item As Quantitative Strategy Developer for around 3.5 years.
  \item As Data Scientist for around 1 year.
  \item As Software Engineer for around 6 months.
\end{itemize}

\section{Personal Details}

\begin{tabular}{ll}
  D.O.B. & 09\textsuperscript{th} December 1993 \\
  Sex & Male \\
  Nationality & Indian
\end{tabular}

\end{document}
