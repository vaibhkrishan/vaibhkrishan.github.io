\documentclass[a4paper,10pt]{article}

\RequirePackage{color,graphicx}
\usepackage{url,parskip}
\usepackage[usenames,dvipsnames]{xcolor}
\usepackage[big]{layaureo}        %better formatting of the A4 page
% an alternative to Layaureo can be ** \usepackage{fullpage} **
\usepackage{supertabular}         %for Grades
\usepackage{titlesec}         %custom \section

\usepackage{hyperref}
\definecolor{linkcolour}{rgb}{0,0.2,0.6}
\hypersetup{colorlinks,breaklinks,urlcolor=linkcolour, linkcolor=linkcolour}

%CV Sections inspired by: 
%http://stefano.italians.nl/archives/26
\titleformat{\section}{\Large\scshape\raggedright}{}{0em}{}[\titlerule]
\titlespacing{\section}{0pt}{1em}{1em}

\begin{document}

\par{\centering
{\Huge Vaibhav \textsc{Krishan}
}\bigskip\par}

\begin{tabular}{rl}
  Phone: & +91-8454937541 \\
  Personal email: & \href{mailto:vaibhkrishan@gmail.com}{vaibhkrishan@gmail.com} \\ 
  Official emails: & \href{mailto:vaibhavk@imsc.res.in}{vaibhavk@imsc.res.in}\\
  Profile pages: & \href{https://vaibhkrishan.github.io}{Homepage}, 
  \href{https://dblp.org/pid/187/8296.html}{DBLP},
  \href{https://scholar.google.com/citations?user=dVMC44EAAAAJ&hl=en}{Google Scholar},
  \href{https://eccc.weizmann.ac.il/author/1412/">ECCC (IIT Bombay account)</a>, <a href="https://eccc.weizmann.ac.il/author/985/}{ECCC (Personal account)},
  \href{https://orcid.org/0009-0000-0335-1963}{ORCID}
\end{tabular}

\section{Research Positions}

\begin{tabular}{p{3.5cm}ll}
  {\bf PDF (TCS)} & The Institute of Mathematical Sciences & July 2024-Present
\end{tabular}

\section{Education}

\begin{tabular}{p{3.5cm}ll}
  {\bf Ph.D.} & Indian Institute of Technology Bombay & July 2017-May 2024 \\
  \textsc{Computer Science} & Advised by Prof. Sundar Vishwanathan & \\
  \textsc{and Engineering} & and Prof. Nutan Limaye & \\
  \\
  {\bf B.Tech.} & Indian Institute of Technology Bombay & July 2009-May 2013 \\
  \textsc{Computer Science} & CPI: 7.59/10 & \\
  \textsc{and Engineering} & & \\
\end{tabular}

\section{Journal Publications}

\begin{tabular}{p{2.5cm}|p{11cm}l}
  \href{https://link.springer.com/article/10.1007/s00453-021-00915-7}{{\bf Algorithmica}} & A \#SAT Algorithm for Small Constant-depth Circuits with PTF gates \\
  & with S. Bajpai, D. Kush, N. Limaye and S. Srinivasan \\
  & Algorithmica 84, 1132-1162 (2022).
\end{tabular}

\section{Conference Publications}

\begin{tabular}{p{2.5cm}|p{11cm}l}
  \href{https://eccc.weizmann.ac.il/report/2024/074/#revision1}{{\bf ITCS 2026}} & Lower Bounds and Separations for Torus Polynomials \\
                                                                  & with S. Vishwanathan \\
                                                                  & The 17\textsuperscript{th} Innovations in Theoretical Computer Science Conference, 2026. \\
  \\
  \href{https://link.springer.com/chapter/10.1007/978-3-030-79416-3_15}{{\bf CSR 2021}} & Upper Bound for Torus Polynomials \\
  & The 16th International Computer Science Symposium in Russia, 2021. \\
  \\
  \href{https://drops.dagstuhl.de/opus/volltexte/2018/10101/}{{\bf ITCS 2019}} & A \#SAT Algorithm for Small Constant-depth Circuits with PTF gates \\
  & with S. Bajpai, D. Kush, N. Limaye and S. Srinivasan \\
  & The 10\textsuperscript{th} Innovations in Theoretical Computer Science Conference, 2019.
\end{tabular}

\section{Preprints}

\begin{tabular}{p{2.5cm}|p{11cm}l}
  \href{https://eccc.weizmann.ac.il/report/2016/155/}{{\bf ECCC}} & Isolation Lemma for Directed Reachability and NL vs. L
  \\
  & with N. Limaye
\end{tabular}

\section{Talks and Presentations}

\begin{tabular}{p{2.5cm}|p{11cm}l}
  \href{https://vaibhkrishan.github.io/files/slides/toruslb.pdf}{Presentation} & Lower Bounds and Separations for Torus Polynomials \\
  & ITCS 2026. \\
  \href{https://youtu.be/lJ5oofcZKhQ}{Video} & Lower Bounds and Separations for Torus Polynomials \\
  & ITCS 2026. \\
  \href{https://www.youtube.com/watch?v=SQCH5v0yytw}{Lecture} & Polynomials and Computation \\
                                                              & Foundational Lecture Series on TCS, IMSc, 2025. \\
  \href{https://vaibhkrishan.github.io/files/slides/torus.pdf}{Presentation} & Upper Bound for Torus Polynomials \\
  & CSR 2021. (online) \\
  \href{https://vaibhkrishan.github.io/files/slides/ptf.pdf}{Presentation} & A \#SAT Algorithm for Small Constant-depth Circuits with PTF gates \\
  and \href{https://vaibhkrishan.github.io/files/slides/poster.pdf}{Poster} & ITCS 2019.
\end{tabular}

\section{Service and Engagement}
\begin{tabular}{p{2.5cm}|p{11cm}l}
  Sub-reviewer & Computational Complexity Conference (CCC) 2024. \\
  Sub-reviewer & Foundations of Software Technology and Theoretical Computer Science (FSTTCS) 2024 \\
  Volunteer & FSTTCS 2019
\end{tabular}

\section{Teaching Assistance}

\begin{itemize}
  \item Awarded best TA of the semester for CS 101.
  \item Awarded best TA of the month for Automata Theory.
  \item TA duty for complexity theory twice, design and analysis of algorithms, combinatorics, and theoretical machine learning.
\end{itemize}

\section{Professional Experience}

\begin{itemize}
  \item Freelance work in quantitative trading for multiple clients since May `22.
  \item Quantitative Research Consultant for AlphaGrep, and for 5-Swans, during official break from PhD June `21-May `22.
  \item As Quantitative Developer in AlphaGrep from June `16-July 17 and Jan `14-April `15.
  \item As Data Scientist in Housing.com from May `15-May `16.
  \item As Software Engineer in Samsung from July `13-Dec `13.
\end{itemize}

\section{Personal Details}

\begin{tabular}{ll}
  D.O.B. & 09\textsuperscript{th} December 1993 \\
  Sex & Male \\
  Nationality & Indian
\end{tabular}

\end{document}
